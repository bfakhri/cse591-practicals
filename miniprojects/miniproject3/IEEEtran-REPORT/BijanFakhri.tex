



\documentclass[conference,compsoc]{IEEEtran}









% *** CITATION PACKAGES ***
%
\ifCLASSOPTIONcompsoc
  % IEEE Computer Society needs nocompress option
  % requires cite.sty v4.0 or later (November 2003)
  \usepackage[nocompress]{cite}
\else
  % normal IEEE
  \usepackage{cite}
\fi


% *** GRAPHICS RELATED PACKAGES ***
%
\ifCLASSINFOpdf
  % \usepackage[pdftex]{graphicx}
  % declare the path(s) where your graphic files are
  % \graphicspath{{../pdf/}{../jpeg/}}
  % and their extensions so you won't have to specify these with
  % every instance of \includegraphics
  % \DeclareGraphicsExtensions{.pdf,.jpeg,.png}
\else
  % or other class option (dvipsone, dvipdf, if not using dvips). graphicx
  % will default to the driver specified in the system graphics.cfg if no
  % driver is specified.
  % \usepackage[dvips]{graphicx}
  % declare the path(s) where your graphic files are
  % \graphicspath{{../eps/}}
  % and their extensions so you won't have to specify these with
  % every instance of \includegraphics
  % \DeclareGraphicsExtensions{.eps}
\fi




% correct bad hyphenation here
\hyphenation{op-tical net-works semi-conduc-tor}


\begin{document}
%
% paper title
\title{CSE 591 - Introduction to Deep Learning\\Mini Project 3}


% author names and affiliations
% use a multiple column layout for up to three different
% affiliations
\author{\IEEEauthorblockN{Bijan Fakhri}
\IEEEauthorblockA{School of Computing, Informatics and\\Decision Systems Engineering\\
Arizona State University\\
Tempe, Arizona 85044\\
Email: bfakhri@asu.edu}}

% make the title area
\maketitle

% As a general rule, do not put math, special symbols or citations
% in the abstract
\begin{abstract}
Hyperparameter tuning is an important aspect of deep learning (or any machine learning in general). The goal of Mini Project 3 is to become familiar with the Theano toolbox Yann by tuning the hyperparameters of the Yann MLNN tutorial. Hyperparameters were heuristically ordered from greatest to least impactful (in terms of their impact on error rate). The error rate of the tuned network was 98.47\%.
\end{abstract}



\IEEEpeerreviewmaketitle



\section{Introduction}
% no \IEEEPARstart
The parameters in question were regularization, optimization technique, momentum technique, and the learning rate. Below is a table of tunable hyperparameters, their options and values, under the scope of this mini project. 

\renewcommand{\arraystretch}{1.1}
\begin{center}
  \begin{tabular}{ | c | c  | c | }
    \hline
    \textbf{Hyperparameter} & \textbf{Options} & \textbf{Values} \\ \hline
    Regularization & ON & L1 Coeff \\  & OFF & L2 Coeff \\ \hline
    Optimization & RMSProp & \_ \\ & AdaGrad & \\ \hline
    Momentum &  None & StartVal \\ &  Polyak &  EndVal \\ & Nesterov &  EndEpoch \\ \hline
    Learning Rate &   & AnnealingFactor \\ & -- &  FirstEraRate \\ & &  SecondEraRate \\ \hline
  \end{tabular}
\end{center}

 After succesfully installing Yann, the tutorial was run using the default values. The default values gave very good results at \textbf{98.43\%}. The network was then purged of tuned hyperparameters, making the network as simple as possible, creating a good baseline to work with. Training/testing the clean network resulted in an accuracy of \textbf{97.98\%}. This became the baseline. A table of the state of this network is below. 
 
 \renewcommand{\arraystretch}{1.1}
 \begin{center}
   \begin{tabular}{ | c | c | c | }
     \hline
     \textbf{Hyperparameter} & \textbf{Options} & \textbf{Values} \\ \hline
     Regularization & OFF & -- \\ \hline
     Optimization & RMSProp & -- \\ \hline
     Momentum &  None &  -- \\ \hline
     Learning Rate & -- & (0.05, 0.01, 0.001)  \\ \hline
   \end{tabular}
 \end{center}


\section{Tuning}
%
Based on intuition, I ranked the hyperparameters in order of what I though would have the largest effect.
\textit{Regularization} \textit{Optimization} \textit{Momentum} \textit{LearningRate}. Begining with regularization, turning it on and with the default parameters (0.001, 0.001), accuracy was affected very slightly negatively (textbf{97.97\%}). Raising the L2 coefficient to 0.002 raised accuracy to textbf{98.18\%}. Deciding to keep this, I moved onto \textit{Optimization}. Changing it to textit{Adagrad} had the negative effect of reducing accuracy to \textbf{98.11\%}. No further exploration of optimization technique was deemed necessary. 

% use section* for acknowledgment
\ifCLASSOPTIONcompsoc
  % The Computer Society usually uses the plural form
  \section*{Acknowledgments}
\else
  % regular IEEE prefers the singular form
  \section*{Acknowledgment}
\fi


The authors would like to thank...





% trigger a \newpage just before the given reference
% number - used to balance the columns on the last page
% adjust value as needed - may need to be readjusted if
% the document is modified later
%\IEEEtriggeratref{8}
% The "triggered" command can be changed if desired:
%\IEEEtriggercmd{\enlargethispage{-5in}}

% references section

% can use a bibliography generated by BibTeX as a .bbl file
% BibTeX documentation can be easily obtained at:
% http://mirror.ctan.org/biblio/bibtex/contrib/doc/
% The IEEEtran BibTeX style support page is at:
% http://www.michaelshell.org/tex/ieeetran/bibtex/
%\bibliographystyle{IEEEtran}
% argument is your BibTeX string definitions and bibliography database(s)
%\bibliography{IEEEabrv,../bib/paper}
%
% <OR> manually copy in the resultant .bbl file
% set second argument of \begin to the number of references
% (used to reserve space for the reference number labels box)
\begin{thebibliography}{1}

\bibitem{IEEEhowto:kopka}
H.~Kopka and P.~W. Daly, \emph{A Guide to \LaTeX}, 3rd~ed.\hskip 1em plus
  0.5em minus 0.4em\relax Harlow, England: Addison-Wesley, 1999.

\end{thebibliography}




% that's all folks
\end{document}


